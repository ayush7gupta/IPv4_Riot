\subsection*{About}

This is a small test application to see how the lowlevel-\/driver functions of the proprietary nrf24l01p-\/transceiver work. These functions consist of general S\+PI and G\+P\+IO commands, which abstract the driver-\/functions from the used board.

\subsection*{Predefined pin mapping}

Please compare the {\ttfamily tests/driver\+\_\+nrf24l01p\+\_\+lowlevel/\+Makefile} for predefined pin-\/mappings on different boards. (In addition, you also need to connect to 3V and G\+ND)

\subsection*{Usage}

You should be presented with the R\+I\+OT shell, providing you with commands to initialize the transceiver (command\+: {\ttfamily it}), sending one packet (command\+: {\ttfamily send}) or read out and print all registers of the transceiver as binary values (command\+: {\ttfamily prgs}).

\subsubsection*{Procedure}


\begin{DoxyItemize}
\item take two boards and connect a transceiver to each (it should be also possible to use one board with different S\+P\+I-\/ports)
\item depending on your board, you\textquotesingle{}ll maybe also need to connect a U\+A\+R\+T/tty converter
\item build and flash the test-\/program to each
\item open a terminal (e.\+g. pyterm) for each
\item if possible, reset the board by using the reset-\/button. You\textquotesingle{}ll see \char`\"{}\+\_\+\+Welcome to R\+I\+O\+T\+\_\+\char`\"{} etc.
\item type {\ttfamily help} to see the description of the commands
\item initialize both with {\ttfamily it}
\item with one board, send a packet by typing {\ttfamily send}
\item in the next step you can also use {\ttfamily send} to send data in the other direction
\item now you can use send on both boards/transceivers to send messages between them
\end{DoxyItemize}

\subsection*{Expected Results}

After you did all steps described above, you should see that a 32 Byte sequence (numbers from 32...1) has been transferred from one device to the other. This sequence is printed out from the receiver after the receive interrupt occurred and the receive-\/procedure has been made.

After initialization ({\ttfamily it}) you should see the following output\+:


\begin{DoxyCode}
 > it

Init Transceiver

Registering nrf24l01p\_rx\_handler thread...
################## Print Registers ###################
REG\_CONFIG:
0x0 returned: 00111111

REG\_EN\_AA:
0x1 returned: 00000001

REG\_EN\_RXADDR:
0x2 returned: 00000011

REG\_SETUP\_AW:
0x3 returned: 00000011

REG\_SETUP\_RETR:
0x4 returned: 00101111

REG\_RF\_CH:
0x5 returned: 00000101

REG\_RF\_SETUP:
0x6 returned: 00100111

REG\_STATUS:
0x7 returned: 00001110

REG\_OBSERVE\_TX:
0x8 returned: 00000000

REG\_RPD:
0x9 returned: 00000000

REG\_RX\_ADDR\_P0:
0xa returned: e7 e7 e7 e7 e7

REG\_TX\_ADDR:
0x10 returned: e7 e7 e7 e7 e7

REG\_RX\_PW\_P0:
0x11 returned: 00100000

REG\_FIFO\_STATUS:
0x17 returned: 00010001

REG\_DYNPD:
0x1c returned: 00000000

REG\_FEATURE:
0x1d returned: 00000000
\end{DoxyCode}


After the data has been sent ({\ttfamily send}), you should see the following output on the receiver terminal\+: 
\begin{DoxyCode}
In HW cb
nrf24l01p\_rx\_handler got a message: Received packet.
32 31 30 29 28 27 26 25 24 23 22 21 20 19 18 17 16 15 14 13 12 11 10 9 8 7 6 5 4 3 2 1
\end{DoxyCode}


\subsection*{Compatibility with S\+I24\+R1 or other N\+R\+F24l01p clones}

C\+RC and Auto-\/\+A\+CK should be disabled. Use 
\begin{DoxyCode}
nrf24l01p\_disable\_all\_auto\_ack(&nrf24l01p\_0);
nrf24l01p\_disable\_crc(&nrf24l01p\_0);
\end{DoxyCode}
 after {\ttfamily nrf24l01p\+\_\+init(\&nrf24l01p\+\_\+0, S\+P\+I\+\_\+\+P\+O\+R\+T, C\+E\+\_\+\+P\+I\+N, C\+S\+\_\+\+P\+I\+N, I\+R\+Q\+\_\+\+P\+I\+N) $<$ 0)} 