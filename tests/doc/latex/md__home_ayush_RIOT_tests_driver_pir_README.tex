This is a manual test application for the P\+IR motion sensor driver.

In order to build this application, you need to add the board to the Makefile\textquotesingle{}s {\ttfamily W\+H\+I\+T\+E\+L\+I\+ST} first and define a pin mapping (see below).

\section*{Usage}

There are two ways to test this. You can either actively poll the sensor state, or you can register a thread which receives messages for state changes.

\subsection*{Interrupt driven}

Connect the sensor\textquotesingle{}s \char`\"{}out\char`\"{} pin to a G\+P\+IO of your board that can be configured to create interrupts. Compile and flash this test application like\+: \begin{DoxyVerb}export BOARD=your_board
export PIR_GPIO=name_of_your_pin
make clean
make all-interrupt
make flash
\end{DoxyVerb}


The output should look like\+: \begin{DoxyVerb}kernel_init(): jumping into first task...

PIR motion sensor test application

Initializing PIR sensor at GPIO_8... [OK]

Registering PIR handler thread...     [OK]
PIR handler got a message: the movement has ceased.
PIR handler got a message: something started moving.
PIR handler got a message: the movement has ceased.
\end{DoxyVerb}


\subsection*{Polling Mode}

Connect the sensor\textquotesingle{}s \char`\"{}out\char`\"{} pin to any G\+P\+IO pin of you board. Compile and flash this test application like\+: \begin{DoxyVerb}export BOARD=your_board
export PIR_GPIO=name_of_your_pin
make clean
make all-polling
make flash
\end{DoxyVerb}


The output should look like this\+: \begin{DoxyVerb}kernel_init(): jumping into first task...
PIR motion sensor test application

Initializing PIR sensor at GPIO_10... [OK]

Printing sensor state every second.
Status: lo
...
Status: lo
Status: hi
...\end{DoxyVerb}
 