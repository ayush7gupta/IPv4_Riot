This is a manual test application for the S\+X127X radio driver.

This test application uses the default pin configuration provided by the driver implementation and that matches the ST Nucleo 64 pins layout. It is best to use \href{https://developer.mbed.org/components/SX1272MB2xAS/}{\tt S\+X1272} or \href{https://developer.mbed.org/components/SX1276MB1xAS/}{\tt S\+X1276} mbed modules with nucleo boards or the all-\/in-\/one \href{http://www.st.com/en/evaluation-tools/p-nucleo-lrwan1.html}{\tt ST P-\/\+N\+U\+C\+L\+E\+O-\/\+L\+R\+W\+A\+N1 Lo\+Ra kit}.

If you have other hardware (boards, Semtech based Lo\+Ra module), you can adapt the configuration to your needs by copying an adapted version of {\ttfamily drivers/sx127x/include/sx127x\+\_\+params.\+h} file to your application directory.

By default the application builds the S\+X1276 version of the driver. If you want to use this application with a S\+X1272 module, set the variable {\ttfamily D\+R\+I\+V\+ER} in the application \mbox{[}Makefile\mbox{]}(Makefile)\+: 
\begin{DoxyCode}
DRIVER = sx1272
\end{DoxyCode}
 instead of 
\begin{DoxyCode}
DRIVER = sx1276
\end{DoxyCode}
 You can also pass {\ttfamily D\+R\+I\+V\+ER} when building the application\+: 
\begin{DoxyCode}
$ make BOARD=nucleo-l073 DRIVER=sx1272 -C tests/drivers\_sx127x flash term
\end{DoxyCode}


\subsection*{Usage}

This test application provides low level shell commands to interact with the S\+X1272/\+S\+X1276 modules.

Once the board is flashed and you are connected via serial to the shell, use the {\ttfamily help} command to display the available commands\+: 
\begin{DoxyCode}
> help
help
Command              Description
---------------------------------------
setup                Initialize LoRa modulation settings
random               Get random number from sx127x
channel              Get/Set channel frequency (in Hz)
register             Get/Set value(s) of registers of sx127x
send                 Send raw payload string
listen               Start raw payload listener
reboot               Reboot the node
ps                   Prints information about running threads.
\end{DoxyCode}


Once the board is booted, use {\ttfamily setup} to configure the basic Lo\+Ra settings\+:
\begin{DoxyItemize}
\item Bandwidth\+: 125k\+Hz, 250k\+Hz or 500k\+Hz
\item Spreading factor\+: between 7 and 12
\item Code rate\+: between 5 and 8
\end{DoxyItemize}

Example\+: 
\begin{DoxyCode}
> setup 125 12 5
setup: setting 125KHz bandwidth
[Info] setup: configuration set with success
\end{DoxyCode}


All values are supported by both S\+X1272 and S\+X1276.

The {\ttfamily random} command use the Semtech to generate a random integer value.

Example\+: 
\begin{DoxyCode}
> random
random: number from sx127x: 2339536315
> random
random: number from sx127x: 863363442
\end{DoxyCode}


The {\ttfamily channel} command allows to change/query the RF frequency channel. The default is 868\+M\+Hz for Europe, change to 915\+M\+Hz for America. The frequency is given/returned in Hz.

Example\+: 
\begin{DoxyCode}
> channel set 868000000
New channel set
> channel get
Channel: 868000000
\end{DoxyCode}


The {\ttfamily register} command allows to get/set the content of the module registers. Example\+: 
\begin{DoxyCode}
> register get all
- listing all registers -
Reg   0  1  2  3  4  5  6  7  8  9  A  B  C  D  E  F
0x00 00 80 1A 0B 00 52 D9 00 00 0F 08 2B 00 4D 80 00
0x10 00 FF 00 00 00 00 00 00 00 00 00 00 00 72 C4 0A
0x20 00 08 01 FF 00 00 08 00 00 00 00 00 00 50 14 40
0x30 00 03 05 67 1C 0A 00 0A 42 12 64 19 01 A1 00 00
0x40 00 10 22 13 0E 5B DB 24 0E 81 3A 2E 00 03 00 00
0x50 00 00 04 23 01 24 3F B0 09 05 84 0B D0 0B D0 32
0x60 2B 14 00 00 10 00 00 00 0F E0 00 0C F6 10 1D 07
0x70 00 5C 00 00 00 00 00 00 00 00 00 00 00 00 00 00
- done -
\end{DoxyCode}


Use the {\ttfamily send} and {\ttfamily receive} commands in order to exchange messages between several modules. You need first to ensure that all modules are configured the same\+: use {\ttfamily setup} and {\ttfamily channel} commands to configure them correctly.

Assuming you have 2 modules, one listening and one sending messages, do the following\+:
\begin{DoxyItemize}
\item On listening module\+: 
\begin{DoxyCode}
> setup 125 12 5
setup: setting 125KHz bandwidth
[Info] setup: configuration set with success
> channel set 868000000
New channel set
> listen
Listen mode set
\end{DoxyCode}

\item On sending module\+: 
\begin{DoxyCode}
> setup 125 12 5
setup: setting 125KHz bandwidth
[Info] setup: configuration set with success
> channel set 868000000
New channel set
> send This\(\backslash\) is\(\backslash\) RIOT!
\end{DoxyCode}

\end{DoxyItemize}

On the listening module, the message is captured\+: 
\begin{DoxyCode}
\{Payload: "This is RIOT!" (13 bytes), RSSI: 103, SNR: 240\}
\end{DoxyCode}
 