This application is a showcase for testing L\+W\+M\+AC communications. Using it for your board, you should be able to interactively use any hardware that is supported for communications among devices based on L\+W\+M\+AC.

\section*{Usage }

Build, flash and start the application\+: 
\begin{DoxyCode}
export BOARD=your\_board
make
make flash
make term
\end{DoxyCode}


The {\ttfamily term} make target starts a terminal emulator for your board. It connects to a default port so you can interact with the shell, usually that is {\ttfamily /dev/tty\+U\+S\+B0}. If your port is named differently, the {\ttfamily P\+O\+RT=/dev/yourport} variable can be used to override this.

\section*{Example output }

The {\ttfamily ifconfig} command will help you to configure all available network interfaces. On an samr21 board it will print something like\+: 
\begin{DoxyCode}
2015-09-16 16:58:37,762 - INFO # ifconfig
2015-09-16 16:58:37,766 - INFO # Iface  4   HWaddr: 9e:72  Channel: 26  NID: 0x23  TX-Power: 0dBm  State:
       IDLE CSMA Retries: 4
2015-09-16 16:58:37,768 - INFO #            Long HWaddr: 36:32:48:33:46:da:9e:72
2015-09-16 16:58:37,769 - INFO #            AUTOACK  CSMA
2015-09-16 16:58:37,770 - INFO #            Source address length: 2
\end{DoxyCode}


The {\ttfamily txtsnd} command allows you to send a simple string directly over the link layer (here, it is L\+W\+M\+AC) using unicast or broadcast. The application will also automatically print information about any received packet over the serial. This will look like\+: 
\begin{DoxyCode}
2015-09-16 16:59:29,187 - INFO # PKTDUMP: data received:
2015-09-16 16:59:29,189 - INFO # ~~ SNIP  0 - size:  28 byte, type:
NETTYPE\_UNDEF (0)
2015-09-16 16:59:29,190 - INFO # 000000 7b 3b 3a 02 85 00 e7 fb 00 00 00 00 01
02 5a 55
2015-09-16 16:59:29,192 - INFO # 000010 40 42 3e 62 f2 1a 00 00 00 00 00 00
2015-09-16 16:59:29,194 - INFO # ~~ SNIP  1 - size:  18 byte, type:
NETTYPE\_NETIF (-1)
2015-09-16 16:59:29,195 - INFO # if\_pid: 4  rssi: 49  lqi: 78
2015-09-16 16:59:29,196 - INFO # src\_l2addr: 5a:55:40:42:3e:62:f2:1a
2015-09-16 16:59:29,197 - INFO # dst\_l2addr: ff:ff
2015-09-16 16:59:29,198 - INFO # ~~ PKT    -  2 snips, total size:  46 byte
\end{DoxyCode}
 